% !TEX program = xelatex
\documentclass{ctexart}
\usepackage{mathtools}
\usepackage{ctex}
\usepackage{amsmath,amsthm,amssymb}
\usepackage{hyperref}
\usepackage{geometry}
\usepackage{appendix}
\usepackage{multicol}
%\usepackage{ntheorem}
\newtheorem{attention}{注意事项}
\newtheorem{definition}{定义}[section]
\newtheorem{theorem}{定理}[section]
\newtheorem{inference}{推论}[section]
\newtheorem{question}{习题}
\newtheorem{solution}{解答}
\usepackage{graphicx}%插入图片的宏包
\usepackage{float}%设置图片浮动位置的宏包
\usepackage{subfigure}%插入多图时用子图显示的宏包
\usepackage{color}

\newcommand{\upcite}[1]{\textsuperscript{\textsuperscript{\cite{#1}}}}
\newcommand*{\dif}{\mathop{}\!\mathrm{d}}
\author{张阳}
\date{\today}
\title{常微分方程}
\numberwithin{equation}{subsection}
\geometry{left=2.0cm, right=2.0cm, top=2.5cm, bottom=2.5cm}
\begin{document}
\maketitle
\tableofcontents
\newpage
\section{一阶微分方程的初等解法}
\subsection{变量分离方程与变量变换}
\subsubsection{可分离变量的微分方程}
   \begin{equation}
        \label{eq:可分离变量的微分方程}
    \frac{{\dif y}}{{\dif x}} = f(x)\varphi(y)
   \end{equation}
   其中$f(x),\varphi (y)$分别是 $x$ 与 $y$ 的已知连续函数.
\begin{itemize}
\item 情形一:如果${\varphi (y) \ne 0}$
            
 原式可化为:
$$\frac{{\dif y}}{{\varphi (y)}} = f(x)dx$$

则方程 \eqref{eq:可分离变量的微分方程} 的通解为:
\[G(y)=F(x)+C\]
其中 $G(y)$,$F(x)$ 分别表示 $\frac{1}{{\phi (y)}}$ 和 $f(x)$ 的某一个原函数
\item 情形二:如果存在 ${\widetilde{y}}$ 使得 ${\varphi (\widetilde{y}) = 0}$,那么${\widetilde{y}}$就是方程 \eqref{eq:可分离变量的微分方程} 的解
\end{itemize}
\subsubsection{齐次方程}
\begin{equation}
        \label{eq:齐次方程}
        \frac{{\dif y}}{{\dif x}} = g(\frac{y}{x})
    \end{equation}
令 $\frac{y}{x} = u$,则 \eqref{eq:齐次方程} 可化为:
\[x\frac{\dif u}{{\dif x}} + u = g(u)\]
整理,得:
\[\frac{{\dif u}}{{\dif x}} = \frac{1}{x} \cdot (g(u) - u)\]
至此,问题转化为 \eqref{eq:可分离变量的微分方程}

\subsubsection{可化为齐次方程的类型}
\begin{equation}
\label{eq:可化为齐次方程的类型}
\frac{{\dif y}}{{\dif x}} = \frac{{{a_1}x + {b_1}y + {c_1}}}{{{a_2}x + {b_2}y + {c_2}}}
\end{equation}
            其中${a_i},{b_i},{c_i},i = 1,2$均为常数,且${c_1},{c_2}$不同时为零.
\begin{itemize}
        \item 情形一:如果$\begin{vmatrix} {{a_1}}&{{b_1}}\\
        {{a_2}}&{{b_2}}\end{vmatrix} =0$    
                            
设 ${a_1} = k{a_2}, {b_1} = k{b_2}$
则原方程可化为:
\[\frac{{\dif y}}{{\dif x}} = \frac{{k({a_2}x + {b_2}y) + {c_1}}}{{{a_2}x + {b_2}y + {c_2}}} = f({\kern 1pt} {a_2}x + {b_2}y)\]

令 $u = {a_2}x + {b_2}y$,带入原方程,整理,得:$$\frac{{\dif u}}{{\dif x}} = {a_2} + {b_2}f(u)$$
至此,问题转化为 \eqref{eq:可分离变量的微分方程}

\item 情形一:如果$\begin{vmatrix} {{a_1}}&{{b_1}}\\
{{a_2}}&{{b_2}}\end{vmatrix} \ne 0    $

那么方程组
        $$\left\{ \begin{array}{l}
        {a_1}x + {b_1}y + {c_1} = 0\\
        {a_2}x + {b_2}y + {c_2} = 0
        \end{array} \right.$$
        存在唯一解$(\alpha ,\beta )$
        令$$\left\{ \begin{array}{l}
                X = x - \alpha \\
                Y = y - \beta 
        \end{array} \right.$$
则原方程可化为
$$\begin{aligned}
\frac{\dif Y}{\dif X}=\frac{\dif y}{\dif x} &=\frac{a_{1}(X+\alpha)+b_{1}(Y+\beta)+c_{1}}{a_{2}(X+\alpha)+b_{2}(Y+\beta)+c_{2}} \\
&=\frac{a_{1} X+b_{1} Y+\left(a_{1} \alpha+b_{1} \beta+c_{1}\right)}{a_{2} X+b_{2} Y+\left(a_{2} \alpha+b_{2} \beta+c_{2}\right)}\\
 &=\frac{a_{1} X+b_{1} Y}{a_{2} X+b_{2} Y}
\end{aligned}$$
至此,问题转化为(1.2)
\end{itemize}  
\subsection{线性方程与常数变易法}
\subsubsection{一阶线性齐次微分方程}
即为 \eqref{eq:可分离变量的微分方程}
\subsubsection{一阶线性非齐次微分方程}
\begin{equation} 
\label{eq:一阶线性非齐次微分方程}
y' = P(x)y + Q(x)
\end{equation}
使用常数变易法,假设 $y = c(x){e^{\int {P(x)\dif x} }}$是方程的解

将上式带入原方程,得:
        $$c'(x){e^{\int {P(x)dx} }} + c(x){e^{\int {P(x)dx} }}P(x) = P(x)c(x){e^{\int {P(x)dx} }} + Q(x)$$
        即:
        $$c'(x) = Q(x){e^{ - \int {P(x)\dif x} }}$$
        积分,得:
        $$c(x) = \int {Q(x){e^{ - \int {p(x)\dif x} }}\dif x + c} $$
        将 $c(x)$ 代入原式,得方程的
        \textcolor[rgb]{1,0,0}{通解:$$y = {e^{\int {P(x)dx} }}\left[\int {Q(x){e^{ - \int {P(x)dx} }}dx + c} \right]$$} 
        
\subsubsection{伯努利方程}
\begin{equation} 
        \label{eq:伯努利方程}
        y' = P(x)y + Q(x){y^n}
\end{equation}

方程左右两边同乘 $y^{- n}$,得:
\[{y^{ - n}}y' = P(x){y^{1 - n}} + Q(x)\]

令 $z = {y^{1 - n}}$,带入上式,得:

\[\frac{1}{{1 - n}}\frac{{\dif z}}{{\dif x}} = P(x)z + Q(x)\]

整理,得:

\[\frac{{\dif z}}{{\dif x}} = (1 - n)P(x)z + (1 - n)Q(x)\]

至此,问题转化为 \eqref{eq:一阶线性非齐次微分方程}

\textcolor[rgb]{1,0,0}{通解为:
\[
    {y^{1 - n}} = {e^{\int {(1 - n)P(x)\dif x} }}\left[\int {(1 - n)Q(x){e^{ - \int {(1 - n)P(x)\dif x} }}\dif x + c} \right]
\]} 

\subsection{恰当方程与积分因子}
\subsubsection{恰当方程}
\begin{equation}  
        \label{eq:恰当方程}
        M(x,y)dx + N(x,y)dy = 0
\end{equation}

其中,

\[\frac{{\partial M(x,y)}}{{\partial y}} = \frac{{\partial N(x,y)}}{{\partial x}} \]

 
\begin{itemize}
\item 解法1:不定积分法:
\textcolor[rgb]{1,0,0}{求\[u(x,y) = \int {M(x,y)dx + \phi (y)} \]
由 $\frac{{\partial u}}{{\partial y}} = N(x,y)$ 求得 $\phi (y)$}
\item 解法2:分组凑微法
        采用``分项组合''的方法,把本身已构成全微分的项分出来,再把剩余的项凑成全微分.
需要熟记的简单二元函数的全微分:
\begin{equation}
\label{eq:二元函数的全微分}
\left\{
\begin{aligned}
        &y\dif x + x\dif y = \dif (xy)\\
        &\frac{{y\dif x - x\dif y}}{{{y^2}}} = \dif \left(\frac{x}{y}\right)\\
        &\frac{{ - y\dif x + x\dif y}}{{{x^2}}} = \dif \left(\frac{y}{x}\right)\\
        &\frac{{y\dif x - x\dif y}}{{xy}} = \dif \left(\ln |\frac{x}{y}|\right)\\
        &\frac{{y\dif x - x\dif y}}{{{x^2} + {y^2}}} = \dif \left(\arctan \frac{x}{y}\right)\\
        &\frac{{y\dif x - x\dif y}}{{{x^2} - {y^2}}}=\frac{1}{2}\dif \left(\ln \left| {\frac{{x - y}}{{x + y}}} \right|\right)
\end{aligned}\right.
\end{equation}
\end{itemize}

\textcolor[rgb]{1,0,0}{恰当微分方程\eqref{eq:恰当方程}的通解就是:
\[
\int M(x,y) \dif x+\int [N-\frac{\partial}{\partial y}\int M(x,y) \dif x]\dif y=c
\]}
\subsubsection{积分因子}

\begin{definition}

如果存在连续可微函数 $\mu(x, y)\ne 0$,使得
\[
\mu (x, y)M(x, y)\dif x+ \mu (x, y)N(x, y)\dif y= 0
\]
为恰当方程,则 $\mu(x, y)$ 是方程 $M(x,y)\dif x+N(x,y)\dif y=0$ 的一个积分因子
\end{definition}
\begin{theorem}

微分方程 \eqref{eq:恰当方程} 有一个仅依赖于 $x$ 的积分因子的充要条件是\textcolor[rgb]{1,0,0}{
\[
\psi(x)=\frac{(\frac{\partial M}{\partial y}-\frac{\partial N}{\partial x}) }{N}
\]
仅与 $x$ 有关},这时 \eqref{eq:恰当方程} 的积分因子为\textcolor[rgb]{1,0,0}{
\[
\mu(x)=e^{\int\psi(x)\dif x}
\]}

同理,

微分方程 \eqref{eq:恰当方程} 有一个仅依赖于 $y$ 的积分因子的充要条件是:\textcolor[rgb]{1,0,0}{
\[\varphi(y)=\frac{(\frac{\partial M}{\partial y}-\frac{\partial N}{\partial x}) }{-M}\]
仅与 $y$ 有关},这时 \eqref{eq:恰当方程} 的积分因子为:\textcolor[rgb]{1,0,0}{
\[\mu(y)=e^{\int\varphi(y)\dif y}\]}
\end{theorem}
\subsection{一阶隐式方程与参数表示}
\begin{itemize}
        \item 形如 $y=f(x,y')$ 的方程:
        
        引入参数 $\frac{\dif y}{\dif x}=p$,上式变为 $y=f(x,p)$,两边对 $x$ 求导,得 
        \[p=\frac{\partial f}{\partial x}+\frac{\partial f}{\partial p}\frac{\partial p}{\partial x}\]
        \item $x=f(y,y')$
        
        引入参数 $\frac{dy}{dx}=p$,上式变为 $x=f(y,p)$,两边对 $y$ 求导,得 
        \[\frac{1}{p}=\frac{\partial f}{\partial y}+\frac{\partial f}{\partial p}\frac{\partial p}{\partial y}\]
        \item $F(x,y')=0$
        
        记 $p=y'$,则 $F(x,y')=F(x,p)$ 表示 $Oxp$ 平面上的一条曲线.现在把曲线参数化:
        \[\left\{\begin{aligned}
                x&=\varphi(t)\\
                p&=\varPhi(t)\\
        \end{aligned}\right.\]
        则方程的通解为 
        \[\left\{\begin{aligned}
                x&=\varphi(t)\\
                y&=\int\varPhi(t)\varphi'(t)\dif t +c\\
        \end{aligned}\right.\]      
        \item $F(y,y')=0$
        
        记 $p=y'$,则 $F(y,y')=F(y,p)$ 表示 $Oyp$ 平面上的一条曲线.现在把曲线参数化:
        \[\left\{\begin{aligned}
                y&=\varphi(t)\\
                p&=\varPhi(t)\\
        \end{aligned}\right.\]
        则方程的通解为 
        \[\left\{\begin{aligned}
                x&=\int\frac{\varphi'(t)}{\varPhi(t)}\dif t+c\\
                y&=\varphi(t)\\
        \end{aligned}\right.\] 
\end{itemize}
\section{一阶微分方程的解的存在定理}
\subsection{解的存在唯一性定理}
\begin{definition}
        函数 $f(x,y)$ 称为在 $R$ 上关于 $y$ 满足利普希茨条件,如果存在常数 $L>0$,使得不等式
        \[|f(x,y_1)-f(x,y_2)|\leqslant L\,|y_1-y_2|\]
        对于所有 $(x,y_1),(x,y_2)\in R$ 都成立,$L$ 称为利普希茨常数.
\end{definition}
\begin{theorem}
        \label{theorem:解的存在唯一性定理}
        \begin{equation}
                \label{eq:解的存在唯一性定理}
                \frac{\dif y}{\dif x}=f(x,y)
        \end{equation}
        如果 $f(x,y)$ 在矩形域 
        \[R:|x-x_0|\leqslant a,|y-y_0|\leqslant b\] 
        上连续且关于 $y$ 满足利普希茨条件\footnote{
        在闭矩形域上 $f_y(x,y)$ 存在连续是保证 $f(x,y)$ 对 $y$ 满足利普希茨条件的充分条件.   
        },则方程 \eqref{eq:解的存在唯一性定理} 存在唯一的解 $y=\varphi(x)$,定义于区间 $|x-x_0|\leqslant h$ 上,连续且满足初值条件
        \begin{equation}
                \label{eq:解的存在唯一性定理:初值条件}
                \varphi(x_0)=y_0,
        \end{equation}
        这里 $h=\min(a,\frac{b}{M})$,$M=\max\limits_{(x,y)\in R}|f(x,y)|$
        \begin{attention}     
                定理 \ref{theorem:解的存在唯一性定理} 中的证明思路为:寻找微分方程的解转变为寻找积分方程的解.具体表述如下:

                设 $y=\varphi(x)$ 是方程 \ref{eq:解的存在唯一性定理} 的定义于区间 $x_0\leqslant x_0+h$ 上,满足初值条件 
                \[\varphi(x_0)=y_0\]
                的解,则 $y=\varPhi(x)$ 是积分方程
                \[y=y_0+\int_{x_0}^xf(x,y)\dif x,x_0\leqslant x\leqslant x_0+h \]
                的定义于 $x_0\leqslant x\leqslant x_0+h$ 上的连续解,反之亦然.
        \end{attention}
        \begin{attention}
                \textcolor[rgb]{1,0,0}{皮卡逼近函数序列构造方法如下:
                \[\left\{\begin{aligned}
                &\varphi_0(x)=y_0\\
                &\varphi_n(x)=y_0+\int_{x_0}^xf(\xi,\varphi_{n-1}(\xi))\dif\xi,x_0\leqslant x\leqslant x_0+h\qquad (n=1,2,\cdots)
        \end{aligned}\right.\]   }
        \end{attention}
        \subsection{近似计算和误差估计}
        第 $n$ 次近似解 $\varphi_n(x)$ 和真正解 $\varphi(x)$ 在区间 $|x-x_0|\leqslant h$ 内的误差估计如下:
        \[|\varphi_n(x)-\varphi(x)|\leqslant\frac{ML^n}{(n+1)!}h^{n+1}\]
\end{theorem}
\subsection{解的延拓定理}
\begin{definition}
        假设方程\eqref{eq:解的存在唯一性定理} 右端函数$f(x,y)$在某一区域G内连续,且关于y满足局部的利普希茨条件,即对于区域G内的每一点,有以其为中心的完全含于G内的闭矩形R存在,在R上$f(x,y)$关于y满足利普希茨条件(对于不同的点,域R的大小和常数L可能不同)。
 \end{definition}
\begin{theorem}
        \textcolor[rgb]{1,0,0}{ 如果方程 \eqref{eq:解的存在唯一性定理} 右端的函数 $f(x,y)$ 在有界区域 $G$ 中连续,且在 $G$ 内关于 $y$ 满足局部的利普希茨条件,那么方程 \eqref{eq:解的存在唯一性定理} 的通过 $G$ 内任何一点 $(x_0,y_0)$ 的解 $y=\varphi(x)$ 可以延拓,直到点 $(x,\varphi(x))$ 任意接近区域 $G$ 的边界.以向 $x$ 增大的一方延拓来说,如果 $y=\varphi(x)$ 只能延拓到区间 $x_0\leqslant x\leqslant d$ 上,则当 $x\to d$ 时,$(x,\varphi(x))$ 趋于区域 $G$ 的边界. }
\end{theorem}
\begin{inference}
        如果 $G$ 是无界区域,在上面解的延拓定理的条件下,方程 \eqref{eq:解的存在唯一性定理} 的通过点 $(x_0,y_0)$ 的解 $y=\varphi(x)$ 可以延拓,以向 $x$ 增大的一方的延拓来说,有下面的两种情况:
        \begin{enumerate}
                \item 解 $y=\varphi(x)$ 可以延拓到区间 $[x_0,+\infty)$
                \item 解 $y=\varphi(x)$ 只可以延拓到区间 $[x_0,d)$,其中 $d$ 为有界数,则当 $x\to d$ 时,或者 $y=\varphi(x)$ 无界,或者点 $(x,\varphi(x))$ 趋于区间 $G$ 的边界
                \item 如果函数 $f(x,y)$ 于整个 $Oxy$ 平面上有定义、连续和有界,同时存在关于 $y$ 的一阶连续偏导数,则方程 \eqref{eq:解的存在唯一性定理} 的任一解均可以延拓到区间 $(-\infty<x<+\infty)$
        \end{enumerate}
\end{inference}
\subsection{解对初值的连续性和可微性定理}
\subsubsection{解对初值的连续依赖性定理}
\begin{theorem}
        \textcolor[rgb]{1,0,0}{假设$f(x,y)$于域G内连续且关于y满足局部利普希茨条件,$(x_0,y_0) \in G,y = \varphi (x,x_0,y_0)$ 是方程\eqref{eq:解的存在唯一性定理} 的满足初值条件$y(x_0)=y_0$的解,它于区间$a \le x \le b$ 上有意义($a \le x_0 \le b$),那么,对任意给定的$\epsilon >0$,必能找到正数$\delta=\delta(\epsilon,a,b)$,使得当\[
            (\bar{x}_0-x_0)^2+(\bar{y}_0-y_0)^2 \le \delta^2    
        \]时,方程\eqref{eq:解的存在唯一性定理}的满足条件$y(\bar{x_0})=\bar{y_0}$的解$y=\varphi(x,\bar{x}_0,\bar{y}_0)$在区间$a \le x \le b$上也有定义,并且\[|\varphi(x,\bar{x}_0,\bar{y}_0)-\varphi(x,x_0,y_0)|<\epsilon,a \le x \le b \] }
\end{theorem}
\subsubsection{解对初值和参数的连续依赖性定理}
\begin{definition}
        \[ G_\lambda : (x,y) \in G, \alpha < \lambda < \beta .\]
\end{definition}
\begin{theorem}
        假设$f(x,y,\lambda)$于域$G_\lambda$内连续,且在$G_\lambda$内关于一致地满足局部利普希茨条件,$(x_0,y_0,\lambda_0) \in G_\lambda,y = \varphi (x,x_0,y_0,\lambda_0)$ 是方程 \eqref{eq:解的存在唯一性定理}$_\lambda$ 通过点$(x_0,y_0)$的解,在区间$a \le x \le b$ 上定义,其中$a \le x_0 \le b$,那么,对任意给定的$\epsilon >0$,必能找到正数$\delta=\delta(\epsilon,a,b)$,使得当\[
            (\bar{x}_0-x_0)^2+(\bar{y}_0-y_0)^2 +(\lambda-\lambda_0)^2 \le \delta^2    
        \]时,方程\eqref{eq:解的存在唯一性定理}$_\lambda$通过点$(x_0,y_0)$的解$y=\varphi(x,\bar{x}_0,\bar{y}_0,\lambda)$在区间$a \le x \le b$上也有定义,并且\[|\varphi(x,\bar{x}_0,\bar{y}_0,\lambda)-\varphi(x,x_0,y_0,\lambda_0)|<\epsilon,a \le x \le b \]
\end{theorem}
\section{高阶微分方程}
\subsection{线性微分方程的解的性质与结构}
\label{section:线性微分方程的解的性质与结构}
\begin{equation}
        \label{eq:高阶非齐次线性微分方程}
        \frac{\dif^nx}{\dif t^n}+a_1(t)\frac{\dif^{n-1}x}{\dif t^{n-1}}+\cdots+a_{n-1}(t)\frac{\dif x}{\dif t}+a_n(t)x=f(t)
\end{equation}
当 $f(t)\equiv0$,非齐次方程\eqref{eq:高阶非齐次线性微分方程} 退化为齐次方程
\begin{equation}
        \label{eq:高阶齐次线性微分方程}
        \frac{\dif^nx}{\dif t^n}+a_1(t)\frac{\dif^{n-1}x}{\dif t^{n-1}}+\cdots+a_{n-1}(t)\frac{\dif x}{\dif t}+a_n(t)x=0
\end{equation}
\begin{theorem}
        如果 $a_i(t)(i=1,2,\cdots,n)$ 及 $f(x)$ 都是区间 $a\leqslant t\leqslant b$ 上连续函数,则对于任一 $t_0\in[a,b]$ 及任意的 $x_0,x_0^{(1)},\cdots,x_0^{(n-1)}$ ,方程 \eqref{eq:高阶非齐次线性微分方程} 存在唯一解 $x=\varphi(t)$,定义于区间 $a\leqslant t\leqslant b$ 上,且满足初值条件
        \begin{equation}
                \label{eq:高阶齐次线性微分方程:初值条件}
                \varphi(t_0)=x_0,\frac{\dif \varphi(t_0)}{\dif t}=x_0^{(1)},\cdots,\frac{\dif^{n-1}\varphi(t_0)}{\dif t^{n-1}}=x_0^{(n-1)}
        \end{equation}
\end{theorem}
\begin{definition}
        由定义在区间 $a\leqslant t\leqslant b$ 上的 $k$ 个可微 $k-1$ 次的函数 $x_1(t),x_2(t),\cdots,x_k(t)$ 所作成的行列式
        \[W[x_1(t),x_2(t),\cdots,x_k(t)]\equiv W(t)=\begin{vmatrix}
                x_1(t)&x_2(t)&\cdots&x_k(t)\\
                x_1'(t)&x_2'(t)&\cdots&x_k'(t)\\
                \vdots&\vdots&&\vdots\\
                x_1^{(k-1)}(t)&x_2^{(k-1)}(t)&\cdots& x_k^{(k-1)}(t)
        \end{vmatrix}.\]
        称为这些函数的{\bf 朗斯基行列式}.
\begin{theorem}
        \label{theorem:朗斯基行列式1}
        若函数 $x_1(t),x_2(t),\cdots,x_n(t)$ 在区间 $a\leqslant t\leqslant b$ 上线性相关,则在 $[a,b]$ 上它们的朗斯基行列式 $W(t)\equiv 0$
\end{theorem}
\begin{theorem}
        \label{theorem:朗斯基行列式2}
        如果方程 \eqref{eq:高阶齐次线性微分方程} 的解 $x_1(t),x_2(t),\cdots,x_n(t)$ 在区间 $a\leqslant t \leqslant b$ 上线性无关,则 $W(t)$ 在这个区间上的任何点上都不等于零,即 $W(t)\ne 0(a\leqslant t\leqslant b).$
\end{theorem}
\begin{attention}
        由定理 \ref{theorem:朗斯基行列式1} 和定理 \ref{theorem:朗斯基行列式2} 可以知道,由 $n$ 阶齐次线性微分方程 \eqref{eq:高阶齐次线性微分方程} 的 $n$ 个解构成的朗斯基行列式或者恒等于零,或者在方程的系数为连续的区间内处处不等于零.
\end{attention}
\end{definition}
\begin{theorem}
        {\bf 叠加原理:}如果 $x_1(t),x_2(t),\cdots,x_k(t)$ 是方程 \eqref{eq:高阶齐次线性微分方程} 的 $k$ 个解,则他们的线性组合 $c_1x_1(t)+c_2x_2(t)+\cdots+c_kx_k(t)$ 也是 \eqref{eq:高阶齐次线性微分方程} 的解,这里 $c_1,c_2,\cdots,c_k$ 为任意常数.

        特别的,如果 $x_1(t),x_2(t),\cdots,x_n(t)$ 是方程 \eqref{eq:高阶齐次线性微分方程} 的 $n$ 个线性无关的解,则他们的线性组合 
        \begin{equation}x=c_1x_1(t)+c_2x_2(t)+\cdots+c_nx_n(t)\end{equation} 是 \eqref{eq:高阶齐次线性微分方程}的通解,且通解包括了方程\eqref{eq:高阶齐次线性微分方程}的所有解,这里 $c_1,c_2,\cdots,c_n$ 为任意常数.

        再特别的,如果 $x(t)$ 是齐次方程 \eqref{eq:高阶齐次线性微分方程} 的通解, $\overline{x}(t)$ 是非齐次方程 \eqref{eq:高阶非齐次线性微分方程} 的一个特解,那么 $\widetilde{x}(t)=x(t)+\overline{x}(t)$ 是非齐次方程的通解.
\end{theorem}
\subsection{一般齐次线性微分方程的解法}
无一般解法.
\subsection{一般的非齐次线性微分方程的解法}
\subsubsection{常数变易法}
\label{section:常数变易法3}
如果齐次方程 \eqref{eq:高阶齐次线性微分方程} 存在解
\[x=c_1x_1(t)+c_2x_2(t)+\cdots+c_nx_n(t)\]
那么非齐次方程 \eqref{eq:常系数非齐次线性微分方程} 存在形如下式的特解:
\[x=c_1(t)x_1(t)+c_2(t)x_2(t)+\cdots+c_n(t)x_n(t)\]
令:
\begin{equation}
        \label{eq:常数变易法}\left\{ \begin{aligned}
        &x_1(t)c_1'(t)+x_2(t)c_2'(t)+\cdots+x_n(t)c_n'(t)=0\\
        &x_1'(t)c_1'(t)+x_2'(t)c_2'(t)+\cdots+x_n'(t)c_n'(t)=0\\
        &\phantom{x_1'(t)c_1'(t)+x_2'(t)c_2'(t)+\,\,}\cdots\\
        &x_1^{(n-2)}(t)c_1'(t)+x_2^{(n-2)}(t)c_2'(t)+\cdots+x_n^{(n-2)}(t)c_n'(t)=0\\
        &x_1^{(n-1)}(t)c_1'(t)+x_2^{(n-1)}(t)c_2'(t)+\cdots+x_n^{(n-1)}(t)c_n'(t)=f(t)
        \end{aligned} \right.\end{equation}
通过求解代数方程 \eqref{eq:常数变易法} 即可求得 $c_1'(x),c_2'(x),\cdots,c_n'(x)$. 进一步积分得到 $c_1(x),c_2(x),\cdots,c_n(x)$,从而得到非齐次方程 \eqref{eq:高阶非齐次线性微分方程} 的一个特解.

\subsection{常系数线性微分方程的解法}
\subsubsection{常系数齐次线性微分方程}
\label{section:常系数齐次线性微分方程}
\begin{equation}
        \label{eq:常系数齐次线性微分方程}
        \frac{\dif^nx}{\dif t^n}+a_1\frac{\dif^{n-1}x}{\dif t^{n-1}}+\cdots+a_{n-1}\frac{\dif x}{\dif t}+a_nx=0
\end{equation}
方程 
\begin{equation}
        \label{eq:常系数齐次线性微分方程的特征方程}
        F(\lambda)=\lambda^n+a_1\lambda^{n-1}+\cdots+a_{n-1}\lambda+a_n=0
\end{equation}
称为方程 \eqref{eq:常系数齐次线性微分方程} 的特征方程,它的根称为特征根.

设$\lambda_1,\lambda_2,\cdots,\lambda_n$是特征方程 \eqref{eq:常系数齐次线性微分方程的特征方程} n个彼此不相等的根 ,相应地方程 \eqref{eq:常系数齐次线性微分方程} 有如下n个解:
\[e^{\lambda_1 t},e^{\lambda_2 t},\cdots,e^{\lambda_n t}\]

方程 \eqref{eq:常系数齐次线性微分方程的特征方程} 的 $k$ 重实根 $x=\lambda_0$ 对应于 \eqref{eq:常系数齐次线性微分方程} 的 $k$ 个解:
\[e^{\lambda t},te^{\lambda t},t^2e^{\lambda t}\cdots,t^{k-1}e^{\lambda t}\]

方程 \eqref{eq:常系数齐次线性微分方程的特征方程} 的 $m$ 重复根 $x=\alpha+\beta i$ 对应于 \eqref{eq:常系数齐次线性微分方程} 的 $2m$ 个解:
\[\begin{aligned}
        &e^{\alpha t}\cos\beta t,&te^{\alpha t}\cos\beta t,&\cdots,&t^{k-1}e^{\alpha t}\cos \beta t,\\
        &e^{\alpha t}\sin\beta t,&te^{\alpha t}\sin\beta t,&\cdots,&t^{k-1}e^{\alpha t}\sin\beta t
\end{aligned}\]



\subsubsection{可化成常系数齐次线性微分方程的方程--欧拉方程}
欧拉方程形式如下
\begin{equation}
        \label{eq:欧拉方程}
        x^n\frac{\dif^n y}{\dif x^n}+a_1x^{n-1}\frac{\dif ^{n-1}y}{\dif x^{n-1}}+\cdots+a_{n-1}x\frac{\dif y}{\dif x}+a_ny=0
\end{equation}

\textcolor[rgb]{1,0,0}{引进自变量的变换\[ x=e^t,t=lnx \]} 


它的特征方程为:

\begin{equation}
        \label{eq:欧拉方程的特征方程}
        K(K-1)\cdots(K-n+1)+a_1K(K-1)\cdots(K-n+2)+\cdots+a_n=0,
\end{equation}
方程 \eqref{eq:欧拉方程的特征方程} 的 $m$ 重实根 $K=K_0$, 对应于欧拉方程 \eqref{eq:欧拉方程} 的 $m$ 个解
\[x^{K_0},x^{K_0}\ln|x|,x^{K_0}\ln^2|x|,\cdots,x^{k_0}\ln^{n-1}|x|\]
方程 \eqref{eq:欧拉方程的特征方程} 的 $m$ 重复根 $K=\alpha+i\beta$,对应于欧拉方程 \eqref{eq:欧拉方程} 的 $2m$ 个解
\begin{align*}
        x^\alpha\cos(\beta\ln|x|),x^\alpha\ln|x|\cos(\beta\ln|x|),x^\alpha\ln^2|x|\cos(\beta\ln|x|)\cdots,x^{\alpha}\ln^{m-1}|x|\cos(\beta\ln|x|)\\
        x^\alpha\sin(\beta\ln|x|),x^\alpha\ln|x|\sin(\beta\ln|x|),x^\alpha\ln^2|x|\sin(\beta\ln|x|)\cdots,x^{\alpha}\ln^{m-1}|x|\sin(\beta\ln|x|)\\
\end{align*}

\subsection{常系数非齐次线性微分方程的解法}
\begin{equation}
        \label{eq:常系数非齐次线性微分方程}
        \frac{\dif^nx}{\dif t^n}+a_1\frac{\dif^{n-1}x}{\dif t^{n-1}}+\cdots+a_{n-1}\frac{\dif x}{\dif t}+a_nx=f(t)
\end{equation}
\ref{section:常系数齐次线性微分方程} 中已经研究了常系数齐次线性微分方程的通解,由线性微分方程的解的结构 \ref{section:线性微分方程的解的性质与结构} 可知,现在只需要求解出非齐次方程 \eqref{eq:常系数非齐次线性微分方程} 的一个特解就能求出非齐次线性方程\eqref{eq:常系数非齐次线性微分方程} 的通解.

下面讨论如何求出非齐次线性方程 \eqref{eq:常系数非齐次线性微分方程} 的一个特解.
\subsubsection{常数变易法}
与上文同理 \ref{section:常数变易法3}
\subsubsection{\textcolor[rgb]{1,0,0}{比较系数法}}

事实上,经过上面的讨论,我们已经研究清楚了常系数非齐次线性微分方程 \eqref{eq:常系数非齐次线性微分方程} 的一般解法,但实际上常数变易法的计算较为复杂.因此,我们特意利用比较系数法研究了以下两种 $f(t)$ 为特殊类型的微分方程.

比较系数法的好处在于不需要通过积分而通过代数的方法即可求得非齐次线性微分方程的通解,即将求解微分方程的问题转化为某一个代数问题来处理,因而比较简便.
\begin{itemize}
        \item 情形一:$f(t)=(b_0t^m+b_1t^{m-1}+\cdots+b_{m-1}t+b_m)e^{\lambda t}$
在此情形下,方程 \eqref{eq:常系数非齐次线性微分方程} 有特解 
                \[x=t^k(B_0t^m+B_1t^{m-1}+\cdots+B_m)e^{\lambda t}\]
其中 $k$ 为特征方程 \eqref{eq:常系数齐次线性微分方程的特征方程} 中的根 $\lambda$ 的重数.\footnote{如果 $\lambda$ 不是方程 \eqref{eq:常系数齐次线性微分方程的特征方程} 的根,$k$ 取0}

        \item 情形二:$f(t)=[A(t)\cos\beta t+B(t)\sin\beta t]e^{\alpha t}$,其中 $\alpha,\beta$ 为常数,而 $A(t),B(t)$ 是带实系数的 $t$ 的多项式,其中一个的次数为 $m$,而另一个的次数不超过 $m$.那么方程 \eqref{eq:常系数非齐次线性微分方程} 有形如下式的特解
        \[x=t^k[P(t)\cos\beta t+Q(t)\sin\beta t]e^{\alpha t}\]
        这里 $k$ 为特征方程 \eqref{eq:常系数齐次线性微分方程的特征方程} 的根 $\alpha+i\beta$ 的重数\footnote{如果 $\alpha+i\beta$ 不是方程 \eqref{eq:常系数齐次线性微分方程的特征方程} 的根,$k$ 取0},而 $P(t),Q(t)$ 均为待定的带实系数的次数不高于 $m$ 的 $t$ 的多项式,可以通过比较系数的办法来确定
\end{itemize}

\subsubsection{\textcolor[rgb]{1,0,0}{拉普拉斯变换}}
设给定微分方程\[\frac{\dif^nx}{\dif t^n}+a_1(t)\frac{\dif^{n-1}x}{\dif t^{n-1}}+\cdots+a_{n-1}(t)\frac{\dif x}{\dif t}+a_n(t)x=f(t) \]
及初值条件\[ x(0)=x_0,{x}'(0)={x_0}',\cdots,x^{(n-1)}(0)=x_0^{(n-1)},  \]其中$a_1,a_2,\cdots,a_n$是常数,而$f(t)$连续且满足原函数的条件。

注意,如果$x(t)$是方程\eqref{eq:常系数非齐次线性微分方程}的任意解,则$x(t)$及其各阶导数$x^{(k)}(k)(k=1,2,\cdot,n)$均是原函数。记\[ 
        F(s)=\mathcal{L}[f(t)] \equiv \int_0^{+\inf} e^{-st} f(t) \dif t ,
        \]
        \[ 
                X(s)=\mathcal{L}[x(t)] \equiv \int_0^{+\inf} e^{-st} x(t) \dif t ,
                \]

那么,按照原函数微分性质有\[\mathcal{L}[{x}'(t)]=sX(s)-x_0,\] \[ \cdots \] 
\[ \mathcal{L}[x^{(n)}(t)]=s^nX(s)-s^{n-1}x_0-s^{n-2}{x_0}'-\cdots-x_0^{(n-1)} ,\]

常用拉普拉斯变换如下:
\begin{table}[H] 
        \begin{center}
            \caption{拉普拉斯变换表}
        \begin{tabular}{|l|l|l|l|}
        \hline
        序号    & 原函数$f(t)$   & 象函数$F(s)$   & $F(s)$ 的定义域   \\ \hline
        1  & 1    & $\frac{1}{s}$ &  Re s>0  \\  \hline
        2    & t & $\frac{1}{s^2}$    & Re s>0 \\ \hline
        3  & $t^{n}(n>-1)$ & $\frac{n!}{s^{n+1}}$     &  Re s>0 \\ \hline
        4   & $e^{zt} $&$\frac{1}{s-z}$        &  Re s>Re z \\ \hline
        5 & $te^{zt}$ & $\frac{1}{(s-z)^2}$     &  Re s>Re z\\ \hline
        6 & $t^ne^{zt}(n>-1)$ & $\frac{n!}{(s-z)^{n+1}}$ & Re s>Re z\\ \hline
        7 & sin $\omega$t &  $\frac{\omega}{s^2+\omega^2}$& Re s>0 \\ \hline
        8 & cos $\omega$t & $\frac{s}{s^2+\omega^2}$& Re s>0\\ \hline
        \end{tabular}
        \end{center}
        \end{table}

\subsection{一些可降阶的方程类型}
\begin{itemize}
        \item 情形一:方程呈形状
        \begin{equation}
                F(t,x^{(k)},x^{(k+1)},\cdots,x^{(n)})=0(1\leqslant k\leqslant n)
        \end{equation}
        只需令 $x^{(k)}=y$,则方程降为关于 $y$ 的 $n-k$ 阶方程
        \begin{equation}
                F(t,y,y',\cdots,y^{(n-k)})=0
        \end{equation}
        \item 情形二:方程呈形状 
        \begin{equation}
                F(x,x',\cdots,x^{(n)})=0
        \end{equation}
        只需令 $x'=y$,并以它为新未知函数,而视 $x$ 为新自变量,那么就有: $x'=y,x''=\frac{\dif y}{\dif t}=\frac{\dif y}{\dif x}x'=y\frac{\dif y}{\dif x},\cdots$ 因此方程就可降低一阶,化成
        \begin{equation}
                G(x,y,\frac{\dif y}{\dif x},\cdots,\frac{\dif^{n-1}y}{\dif x^{n-1}})=0
        \end{equation}

        \item 情形三:齐次线性微分方程
        \begin{equation}
                \frac{\dif ^nx}{\dif t^n}+a_1(t)\frac{\dif ^{n-1}x}{\dif t^{n-1}}+\cdots+a_n(t)x=0
        \end{equation}
        如果知道方程的一个解 $x_1(t)$,令 $z=y'=(\frac{x}{x_1})$,或 $x=x_1\int z \dif t$,方程可化成 $n-1$ 阶的,关于 $z$ 的齐次线性微分方程.

        特别的,对于二阶齐次线性微分方程
        \[\frac{\dif ^2x}{\dif t^2}+p(t)\frac{\dif x}{\dif t}+q(t)x=0\]
        如果知道了方程的一个解 $x_1(t)$,那么方程的通解为:
        \[x=x_1\left[c_1+c_2\int\frac{1}{x_1^2}e^{-\int p(t)\dif t}\dif t\right]\]
        其中 $c_1,c_2$ 为任意常数.

\end{itemize}




\section{线性微分方程组}
\subsection{研究对象}
非齐次线性微分方程组
\begin{equation}
        \label{eq:一般的一阶线性微分方程组}
        \boldsymbol{x}'=\boldsymbol{A(t)x}+\boldsymbol{f(t)}
\end{equation}

齐次线性微分方程组
\begin{equation}
        \label{eq:一阶线性齐次微分方程组}
        \boldsymbol{x}'=\boldsymbol{A(t)x}
\end{equation}

非齐次线性常系数微分方程组
\begin{equation}
        \label{eq:一阶线性非齐次常系数微分方程组}
        \boldsymbol{x}'=\boldsymbol{Ax}+\boldsymbol{f(t)}
\end{equation}

齐次线性常系数微分方程组
\begin{equation}
        \label{eq:一阶线性齐次常系数微分方程组}
        \boldsymbol{x}'=\boldsymbol{Ax}
\end{equation}

研究的难度在逐步下降,其中方程 \eqref{eq:一阶线性非齐次常系数微分方程组} \eqref{eq:一阶线性齐次常系数微分方程组}  我们可以得到一般性的解法.
\subsection{一些定义}

\begin{definition}朗斯基行列式:

        设有 $n$ 个定义在区间 $a\leqslant t\leqslant b$ 上的向量函数
        \[\boldsymbol{x}_{1}(t)=\left[\begin{array}{c}x_{11}(t) \\ x_{21}(t) \\ \vdots \\ x_{n 1}(t)\end{array}\right], \boldsymbol{x}_{2}(t)=\left[\begin{array}{c}x_{12}(t) \\ x_{22}(t) \\ \vdots \\ x_{n 2}(t)\end{array}\right], \cdots, \boldsymbol{x}_{n}(t)=\left[\begin{array}{c}x_{1 n}(t) \\ x_{2 n}(t) \\ \vdots \\ x_{n1}(t)\end{array}\right],\]
由这 $n$ 个向量函数构成的行列式
\[ \boldsymbol{W}\left[x_{1}(t), x_{2}(t), \cdots, x_{n}(t)\right]
=\boldsymbol{W}(t)=\begin{vmatrix}x_{11}(t) & x_{12}(t) & \cdots & x_{1 n}(t) \\ x_{21}(t) & x_{22}(t) & \cdots & x_{2 n}(t) \\ \vdots & \vdots & & \vdots \\ x_{n 1}(t) & x_{n 2}(t) & \cdots & x_{n n}(t)\end{vmatrix}\]
称为这些向量函数的朗斯基行列式.
\end{definition}
\begin{definition}解矩阵与基解矩阵
        
        如果一个 $n\times n$ 矩阵的每一列都是 \eqref{eq:一般的一阶线性微分方程组} 的解,我们称这个矩阵为 \eqref{eq:一般的一阶线性微分方程组} 的解矩阵.它的列在 $a\leqslant t\leqslant b$ 上是线性无关的解矩阵称为在 $a\leqslant t\leqslant b$ 上 \eqref{eq:一般的一阶线性微分方程组} 的基解矩阵.

\end{definition}
\begin{definition}矩阵指数 $\exp\boldsymbol{A}$

        如果 $A$ 是一个 $n\times n$ 常数矩阵,我们定义矩阵指数 $\exp\boldsymbol{A}$ 为下面的矩阵级数的和:\textcolor[rgb]{1,0,0}{
        \[
\exp \boldsymbol{A}=\sum_{k=0}^{\infty} \frac{\boldsymbol{A}^{k}}{k !}=\boldsymbol{E}+\boldsymbol{A}+\frac{\boldsymbol{A}^{2}}{2 !}+\cdots+\frac{\boldsymbol{A}^{m}}{m !}+\cdots
\]
        }
\end{definition}
\subsection{一些理论}
\begin{theorem}
        每一个 $n$ 阶线性微分方程组可化为 $n$ 个线性微分方程构成的方程组,反之却不成立

$n$阶线性微分方程的初值问题:
\begin{equation}
        \label{eq:n阶线性微分方程的初值问题}
        \left\{\begin{aligned}
                &x^{(n)}+a_1(t)x^{(n-1)}+\cdots+a_{n-1}(t)x'+a_n(t)x=f(t)\\
                &x(t_0)=\eta_1,x'(t_0)=\eta_2,\cdots,x^{(n-1)}(t_0)=\eta_n
        \end{aligned}\right.
\end{equation}
令 \[x_1=x,x_2=x',x_3=x'',\cdots,x_n=x^{n-1}\]
则方程 \eqref{eq:n阶线性微分方程的初值问题} 等价于下列线性微分方程组的初值问题:
\begin{equation}\left\{
        \begin{aligned}
                &\boldsymbol{x}'=\begin{bmatrix}
                        0 & 1 & 0 & \cdots &0 \\
                        0 & 0 & 1 & \cdots & 0\\
                        \vdots &\vdots  &  & \vdots & \\
                        0 & 0 & 0 & \cdots & 1\\
                        -a_n(t) & -a_{n-1}(t) & -a_{n-2}(t) & \cdots &
                       -a_1(t)\end{bmatrix}\boldsymbol{x}+\begin{bmatrix}
                        0\\
                        0\\
                        \vdots\\
                        0\\
                       f(t)
                       \end{bmatrix}\\
                &\boldsymbol{x}(t_0)=\boldsymbol{\eta},
        \end{aligned}
        \right.\end{equation}
\end{theorem}
\begin{theorem}
        如果向量函数 $\boldsymbol{x}_1(t),\boldsymbol{x}_2(t),\cdots,\boldsymbol{x}_n(t)$ 在区间 $a\leqslant t\leqslant b$ 上线性相关,则它们的朗斯基行列式 $W(t)=0,(a\leqslant t\leqslant b)$ 
\end{theorem}
\begin{theorem}
        如果 \eqref{eq:一阶线性齐次微分方程组} 的解 $\boldsymbol{x}_1(t),\boldsymbol{x}_2(t),\cdots,\boldsymbol{x}_n(t)$ 线性无关,那么,它们的朗斯基行列式 $\boldsymbol{W}(t)\ne 0(a\leqslant t\leqslant b)$
\end{theorem}

\begin{theorem}叠加原理
        
        原理形式与高阶线性微分方程的叠加原理相似,此处不再赘述
\end{theorem}

\subsection{求解方法}
\subsubsection{齐次线性微分方程组}
        方程 \eqref{eq:一阶线性齐次微分方程组} 无一般求解方法
\subsubsection{非齐次线性微分方程组}
        如果已知方程 \eqref{eq:一阶线性齐次微分方程组} 对应的齐次微分方程组的通解,可以通过常数变易法求得方程 \eqref{eq:一般的一阶线性微分方程组} 的通解
\begin{theorem}
        如果 $\boldsymbol{\Phi}(t)$ 是 \eqref{eq:一阶线性齐次微分方程组} 的基解矩阵,则向量函数
        \textcolor[rgb]{1,0,0}{
\begin{equation}
        \label{eq:常数变易公式二}
        \boldsymbol{\varphi}(t)=\boldsymbol{\Phi}(t)\boldsymbol{\Phi}^{-1}(t_0)\boldsymbol{\eta}+\boldsymbol{\Phi}(t) \int_{t_{0}}^{t} \boldsymbol{\Phi}^{-1}(s) \boldsymbol{f}(s) \mathrm{d} s
\end{equation} }
是 \eqref{eq:一般的一阶线性微分方程组} 的解,且满足初值条件 $\boldsymbol{\varphi}(t_0)=\boldsymbol{\eta}$ 的解.
        
\end{theorem}
\subsubsection{高阶线性非齐次微分方程}
如果$a_1(t),\cdots,a_n(t),f(t)$是区间$a \le t \le b$上的连续函数,$x_1(t),\cdots,x_n(t)$是区间$a \le t \le b$上的齐次线性微分方程 \[x^{(n)}+a_1(t)x^{(n-1)}+\cdots+a_{n-1}(t)x'+a_n(t)x=0 \]的基本解组,那么,非齐次线性微分方程\[x^{(n)}+a_1(t)x^{(n-1)}+\cdots+a_{n-1}(t)x'+a_n(t)x=f(t) \]的满足初值条件\[\varphi(t_0)=0,{\varphi}'(t_0)=0,\cdots,\varphi^{(n-1)}(t_0)=0,t_0 \in [a,b] \]的解由下面公式给出\begin{equation}
        \label{eq:常数变易公式三}
        \varphi(t)=\sum\limits_{k=1}^n x_k(t) \int_{t_0}^t \{ \frac{\boldsymbol{W_k}[x_1(s),x_2(s),\cdots,x_n(s) ] }{\boldsymbol{W}[x_1(s),x_2(s),\cdots,x_n(s) ]} \} f(s) \mathrm{d} s
\end{equation} 
这里$\boldsymbol{W}[x_1(s),x_2(s),\cdots,x_n(s) ] $是$x_1(s),x_2(s),\cdots,x_n(s) ]$的朗斯基行列式,$\boldsymbol{W_k}[x_1(s),x_2(s),\cdots,x_n(s) ] $是在$\boldsymbol{W}[x_1(s),x_2(s),\cdots,x_n(s) ] $中的第$k$列代以$(0,0,\cdots,0,1)^T$后得到的行列式,而且$x^{(n)}+a_1(t)x^{(n-1)}+\cdots+a_{n-1}(t)x'+a_n(t)x=f(t) $ 的任一解$u(t)$都具有形式 \[ u(t)=c_1x_1(t)+c_2x_2(t)+\cdots+c_nx_n(t)+\varphi(t) \]这里$c_1,c_2,\cdots,c_n$是适当选取的常数。

\textcolor[rgb]{1,0,0}{
当n=2时,公式\eqref{eq:常数变易公式三}变为\begin{equation}
        \label{eq:常数变易公式四}
        \varphi(t)=\int_{t_0}^t  \frac{x_2(t)x_1(s)-x_1(t)x_2(s) }{\boldsymbol{W}[x_1(s),x_2(s)]}  f(s) \mathrm{d} s
\end{equation}
}
\subsubsection{\textcolor[rgb]{1,0,0}{常系数齐次线性微分方程组}}
\begin{itemize}
        \item \textcolor[rgb]{1,0,0}{$\boldsymbol{A}$ 有 $n$ 个线性无关的特征向量}
        \begin{theorem}
        如果矩阵 $A$ 具有 $n$ 个线性无关的特征向量 $\boldsymbol{v_1},\boldsymbol{v_2},\cdots,\boldsymbol{v_n}$ 它们对应的特征值分别为 $\lambda_1,\lambda_2,\cdots,\lambda_n$ (不必各不相同),那么矩阵
        \[\boldsymbol{\Phi}(t)=[e^{\lambda_1t}\boldsymbol{v_1},e^{\lambda_2t}\boldsymbol{v_2},\cdots,e^{\lambda_nt}\boldsymbol{v_n}],-\infty<t<+\infty\]
        是常系数线性微分方程组 \eqref{eq:一阶线性齐次常系数微分方程组} 的一个基解矩阵.
        \end{theorem}
        \item \textcolor[rgb]{1,0,0}{$\boldsymbol{A}$ 是一个任意的 $n\times n$ 矩阵}
        \begin{theorem}
        常系数线性微分方程组 \eqref{eq:一阶线性齐次常系数微分方程组} 具有初值 $\boldsymbol{\varphi}(0)=\boldsymbol{\eta}$,  $\lambda_1,\lambda_2,\cdots,\lambda_n$ 是系数矩阵 $A$ 的 $n_1,n_2,\cdots,n_k$ 重不同特征值,对应的特征向量构成的线性空间记为 $U_1,U_2,\cdots,U_k$.则以下结论成立:
        \begin{itemize}
        \item  $\boldsymbol{\eta}$ 可被分解为 $\boldsymbol{\eta}=\boldsymbol{v}_1+\boldsymbol{v}_2+\cdots+\boldsymbol{v}_k$ 其中 $\boldsymbol{v}_j\in U_j,(j=1,2,\cdots,k)$

         \item 方程 \eqref{eq:一阶线性齐次常系数微分方程组} 满足 $\boldsymbol{\varphi}(0)=\boldsymbol{\eta}$ 的解 $\varphi(t)$ 可以写为\footnote{特别的,当 $A$ 只有一个特征值时,基解矩阵:\[\begin{aligned} \exp \boldsymbol{A} t=\mathrm{e}^{\lambda t} \sum_{i=0}^{n-1} \frac{t^{i}}{i !}(\boldsymbol{A}-\lambda \boldsymbol{E})^{i} \end{aligned}\]
                                
        满足初值条件 $\varphi(0)=\eta$ 的解
        \[\begin{aligned} \varphi(t)=(\exp \boldsymbol{A} t)\boldsymbol{\eta}=\mathrm{e}^{\lambda t} \sum_{i=0}^{n-1} \frac{t^{i}}{i !}(\boldsymbol{A}-\lambda \boldsymbol{E})^{i}\eta \end{aligned}\]}:
        \begin{equation}
        \label{eq:一般矩阵的解法}\boldsymbol{\varphi}(t)=\sum_{j=1}^{k} \mathrm{e}^{\lambda_jt}\left[\sum_{i=0}^{n_{j}-1} \frac{t^{i}}{i !}\left(\boldsymbol{A}-\lambda_{j} \boldsymbol{E}\right)^{i}\right] \boldsymbol{v}_{j}
        \end{equation}
        \item 为了从 \eqref{eq:一般矩阵的解法} 中得到 $\exp\boldsymbol{A}t$ ,只需依次令 $\boldsymbol{\eta}=\boldsymbol{e}_1,\boldsymbol{e}_2,\cdots\boldsymbol{e}_n$ 求得 $n$ 个解,以这 $n$ 个解为列即可得到 $\exp\boldsymbol{A}t$
        \end{itemize}
        \end{theorem}
        \end{itemize}
        \begin{theorem}
                如果已知方程的一个基解矩阵 $\boldsymbol{\Phi}(t)$ 那么 $\exp \boldsymbol{A}t$ 可由下面的公式得到:
                \[\exp \boldsymbol{A}t=\boldsymbol{\Phi}(t)\boldsymbol{\Phi}(0)^{-1}\]
                
                另外, 方程 $\boldsymbol{x}'=\boldsymbol{Ax}$ 满足初值条件 $\boldsymbol{\varphi}(t)=\boldsymbol{\eta}$ 的解 $\boldsymbol{\varphi}(t)$ 可由下列公式计算得到:

                \[\boldsymbol{\varphi}(t)=[\exp \boldsymbol{A}(t-t_0)]\boldsymbol{\eta}\]
        
        \end{theorem}
\subsubsection{常系数非齐次线性微分方程组}
这时,我们有$\boldsymbol{\Phi}^{-1}(s)=exp(-s\boldsymbol{A})$,$\boldsymbol{\Phi}(t)\boldsymbol{\Phi}^{-1}(s)=exp[(t-s)\boldsymbol{A}]$,若初值条件是$\boldsymbol{\varphi}(t_0)=\boldsymbol{\eta}$,则$\boldsymbol{\varphi_k}(t)=exp[(t-t_0)\boldsymbol{A}]\boldsymbol{\eta}$,\eqref{eq:一阶线性非齐次常系数微分方程组}的解就是
\textcolor[rgb]{1,0,0}{\begin{equation}
        \label{eq:常数变易公式五}
        \boldsymbol{\varphi}(t)=exp[(t-t_0)\boldsymbol{A}]\boldsymbol{\eta}+\int_{t_{0}}^{t} exp[(t-s)\boldsymbol{A}] \boldsymbol{f}(s) \mathrm{d} s
\end{equation} }
\begin{thebibliography}{99}
\bibitem[1] .王高雄.周之铭.朱思铭.王寿松.常微分方程第三版.北京:高等教育出版社,2006
\end{thebibliography}
\newpage
\section{后记}
突然很想给常微分写一份复习资料,于是有了此文档.它还有很多不完善的地方,比如一些地方的微分算子仍是斜体,比如一些地方的向量没有用粗体表示,比如结构可能不是很完美.懒得改了,留给以后吧.

如需要latex源码,请联系 CauZhangYang@outlook.com,

如有任何建议,也请联系 CauZhangYang@outlook.com
 
写于 2020年12月24日

\rule[0pt]{0.9\linewidth}{0.09em}

2020-12-27更新日志:优化了微分算子,整理了部分定理的结构,删去了一些不常考的定理,部分内容进行了纠错.

\rule[0pt]{0.9\linewidth}{0.09em}

2021-4.13更新日志:修改了一些错误,删除了一些过于理论的知识,向量改用粗体

\rule[0pt]{0.9\linewidth}{0.09em}

2021-12.27更新日志:增加了部分北京工业大学统计学专业ODE考纲要求知识点,并将部分重要知识点与公式标红。

\quad \quad \quad \quad \quad \quad \quad \quad        ——silvermoonmzq@qq.com

\begin{center}
\Large        希望你喜欢它.
\end{center}
\begin{center}
       本文档是完全免费的,但如果你愿意给我一点点咖啡钱,我会非常开心\,:)
\end{center}
\begin{figure}
      \begin{minipage}[t]{0.5\linewidth}
       \centering
       \includegraphics[width=2.2in]{fig/1.jpg}
      \end{minipage}%
       \begin{minipage}[t]{0.5\linewidth}
       \centering
       \includegraphics[width=2.2in]{fig/2.jpg}
      \end{minipage}
       \end{figure}



\end{document}